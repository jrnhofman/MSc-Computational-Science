%%% DOCUMENT SETUP %%%
\documentclass[11pt,a4paper,onecolumn]{article}
\usepackage[greek,english]{babel}
 
%%% LAYOUT %%%
\usepackage{fullpage}
\usepackage[parfill]{parskip}
\usepackage{multicol}
\usepackage{footnote}

%%% GRAPHICS %%%
\usepackage{graphicx}
\usepackage{color}
\usepackage{graphics}
\usepackage{rotating}
\usepackage{subfig}
\usepackage{amsmath}
\usepackage{amssymb}
\usepackage{amscd}
\usepackage{xfrac}
\usepackage{float}
\usepackage{dsfont}

%%% FONT %%%
\usepackage{ifxetex}
\ifxetex
  \usepackage{fontspec}
    \setmainfont{Linux Libertine O}
  \usepackage{xunicode}
  \usepackage{microtype}
\else
  \usepackage[T1]{fontenc}
  \usepackage[latin1]{inputenc}
  \usepackage{times}
  \usepackage{microtype}
\fi

%%% Coding %%%
\usepackage{listings}
\usepackage{pseudocode}

\begin{document}
\captionsetup{width=0.8\textwidth}
\lstset{language=Python,breaklines=true,backgroundcolor=\color{white},frame=single}
\thispagestyle{empty}

{\Huge Jeroen Hofman \\}
{\Large 2513225 \\
Homework 2 \\
SPF 2012}

\section{Exercise 1}

We are given a process $X$ with a stochastic differential:

\begin{equation}
  dX(t) = \alpha X(t)dt + \sigma X(t)dW(t)
\end{equation}

where $\sigma$ and $\mu$ are adapted processes. Furthermore we have $Z(t) = X^{-1}(T)$. By taking $\mu(t) = \alpha X(t)$ and $\sigma(t) = \sigma X(t)$ we obtain $dX(t) = \mu(t)dt + \sigma(t)dW(t)$. Furthermore we take $f(t,X(t)) = X^{-1}(t)$. Then according to Theorem 4.10 and Lemma 4.11 from Bj\"ork $f$ (and hence $Z$) follows a stochastic differential equation given by:

\begin{equation}
  df(t,X(t)) = \left[ \frac{\partial f}{\partial t} + \mu \frac{\partial f}{\partial x} + \frac{1}{2}\sigma^2\frac{\partial^2f}{\partial x^2}\right]dt + \sigma\frac{\partial f}{\partial x}dW(t)
\end{equation}

By plugging $dX$ in the equation we obtain:

\begin{align}
  dZ &= \left[ \frac{\partial Z}{\partial t} + \mu \frac{\partial Z}{\partial x} + \frac{1}{2}\sigma^2\frac{\partial^2Z}{\partial x^2}\right]dt + \sigma\frac{\partial Z}{\partial x}dW(t) \\
  &= \left[0 + -X^{-2} \alpha X + \frac{1}{2}2X^{-3}\sigma^2X^2\right]dt + (-X^2)\sigma XdW\\
  &= \left(\sigma^2 -\frac{\alpha}{2}\right)X^{-1}dt - \sigma X^{-1} dW \\
  &= \left(\sigma^2 -\frac{\alpha}{2}\right)Zdt - \sigma Z dW
\end{align}

\section{Exercise 2}

Given $m(t) = E[X(t)]$, with $dX(t) = \alpha X(t)dt + \sigma(t)dW(t)$. If we integrate and take the expected value then we obtain:

\begin{align}
  E[X(t)] &= E\left[X_0 + \alpha \int^t_0 X(s)ds + \int^t_0 \sigma(s)dW(s) \right] \\
  &= X_0 + \alpha E\left[\int^t_0 X(s) ds\right] + E\left[\int^t_0 \sigma(s)dW(s)\right]
\end{align}

According to Proposition 4.4 the second integral is zero if:
\begin{itemize}
\item 
  $\sigma(s)$ is an adapted process
\item
  $\int^t_0 E[\sigma^2]ds < \infty$
\end{itemize}

If we assume this to hold, then we are only left with the first integral. For the first integral we can take the expectation inside:

\begin{equation}
  E[X(t)] = X_0 + \alpha \int^t_0 E[X(s)]ds
\end{equation}

If we take the time derivative of this and use that $E[X(t)] = m(t)$, we obtain:

\begin{equation}
  \dot{m}(t) = Z_0 m(t)
\end{equation}

This can be solved to give:

\begin{equation}
  m(t) = Z_0 e^{\alpha t}
\end{equation}


\section{Exercise 3}

We are given a stochastic process:

\begin{equation}
  Z(t) = \frac{W^2(t)}{t}, \; t \geq 1
\end{equation}

According to Lemma 4.9 this process is a martingale iff its Ito differential has no t-dependency. Similarly to Exercise 1 we are given a stochastic differential equation $dX(t) = \mu (t)dt + \sigma (t)dW(t)$ where $\mu = 0$ and $\sigma = 1$. Furthermore we have $f(t,X(t)) = Z(X(t)) = \frac{X^2(t)}{dt}$. We use Theorem 4.10 and Proposition 4.11 just as in Exercise 1 to get: 

\begin{align}
  dZ &= \left[ \frac{\partial Z}{\partial t} + \mu \frac{\partial Z}{\partial x} + \frac{1}{2}\sigma^2\frac{\partial^2Z}{\partial x^2}\right]dt + \sigma\frac{\partial Z}{\partial x}dW(t) \\
  &= \left[\frac{-W^2(t)}{t^2} + 0 + \frac{1}{2}\frac{2}{t} \right]dt + \frac{2W(t)}{t}dW(t) \\
  &= \left[ \frac{-W^2(t)}{t^2} + \frac{1}{t}\right]dt + \frac{2W(t)}{t}dW(t)
\end{align}

Now it seems that this system has a systematic drift and hence is not a martingale in the strict sense. Note however, that the expectation of the dt term is 0 since $E[W^2(s)] = t$, which makes this system behave 'almost' like a martingale.
\end{document}
