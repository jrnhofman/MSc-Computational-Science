%%% DOCUMENT SETUP %%%
\documentclass[11pt,a4paper,onecolumn]{article}
\usepackage[greek,english]{babel}
 
%%% LAYOUT %%%
\usepackage{fullpage}
\usepackage[parfill]{parskip}
\usepackage{multicol}
\usepackage{footnote}

%%% GRAPHICS %%%
\usepackage{graphicx}
\usepackage{color}
\usepackage{graphics}
\usepackage{rotating}
\usepackage{subfig}
\usepackage{amsmath}
\usepackage{amssymb}
\usepackage{amscd}
\usepackage{xfrac}
\usepackage{float}
\usepackage{dsfont}

%%% FONT %%%
\usepackage{ifxetex}
\ifxetex
  \usepackage{fontspec}
    \setmainfont{Linux Libertine O}
  \usepackage{xunicode}
  \usepackage{microtype}
\else
  \usepackage[T1]{fontenc}
  \usepackage[latin1]{inputenc}
  \usepackage{times}
  \usepackage{microtype}
\fi

%%% Coding %%%
\usepackage{listings}
\usepackage{pseudocode}

\begin{document}
\captionsetup{width=0.8\textwidth}
\lstset{language=Python,breaklines=true,backgroundcolor=\color{white},frame=single}
\thispagestyle{empty}

{\Huge Jeroen Hofman \\}
{\Large 2513225 \\
Homework 3 \\
SPF 2012}

\section{Exercise 1}

We have an 'all-or-nothing' contract with delivery date $T$ with strike $K$, where the payoff is 0 if the underlying stock exceeds $L$ during the lifetime of the contract, otherwise it is zero.

If we want to compute the price at $t \in [0,T]$, under the restriction that $S(s) < ln(L)$ for all $s \leq t$, then we can use Proposition 18.4 by noting that we want to compute the probability that the process $X = lnS$ reaches at least a level of $ln(L)$ during the interval $[t,T]$. This is equivalent to 1 minus the probability of having a running maximum of $ln(L)$. This probability is given by:

\begin{equation*}
  (1 - F_{M(T)}(ln(L))) - (1 - F_{M(t)}(ln(L))) = F_{M(t)}(ln(L)) - F_{M(T)}(ln(L))
\end{equation*}

where $F_{M(t)}(ln(L))$ is given by Proposition 18.4.
Since the payoff will be paid out at time $T$ and we want to determine the price at $t$, we have to discount back from $T$ to obtain the price $V(t)$ of the 'all-or-nothing' option at time $t$:

\begin{equation*}
  V(t) = \exp{(r(t-T))}(F_{M(t)}(ln(L)) - F_{M(T)}(ln(L)))
\end{equation*}

where $r$ is the riskfree interest rate as given by the market.

\section{Exercise 2}

Let $T>0$. We have the following problem:

\begin{align*}
  \frac{\partial F(t,x)}{\partial t} + \frac{\sigma^2}{2}\frac{\partial^2 F(t,x)}{\partial x^2} = 0, \; x \in \mathbb{R}, \; t \in [0,T] \\
  F(T,x) = x^3, x \in \mathbb{R}
\end{align*}

We follow example 5.7 from the book. From Proposition 5.5 we have that:

\begin{equation*}
  F(t,x) = E_{t,x}[X^3_T]
\end{equation*}

with

\begin{align*}
  dX_s &= \sigma dW_s \\
  X_t &= x
\end{align*}

giving

\begin{equation*}
  X_T = x + \sigma[W_T - W_t]
\end{equation*}

Now we compute $E_{t,x}[X^3_T]$ by inserting the solution above:

\begin{align*}
  E[X_T^3] &= E[x^3] + E[\sigma^3W^3_{T-t}] + E[3x\sigma^2W_{T-t}^2] + E[3x^2W_{T-t}] \\
  &= x^3 + \sigma^3E[W^3_{T-t}] + 3x\sigma^2E[W_{T-t}] + 3x^2E[W_{T-t}] \\
  &= x^3 + \sigma^3E[W^3_{T-t}] + 3x\sigma^2(T-t) + 0
\end{align*}

We now compute $E[W^3_{T-t}]$ using Ito's lemma with $Z(t,X(t)) = x^3$:

\begin{align*}
  dZ(t) &= 3x^2dW(t) + 0.5 * 6xdW(t) \\
  &= 3W^2(t)dW(t) + 3W(t)dt
\end{align*}

Now integrate and take the expectation:

\begin{align*}
  E[Z(t)] &= 3\int^T_tE[W^2(s)]dW(s) + 3\int_t^TE[W(s)]ds\\
  &= 3\int_t^TsdW(s) + 0\\
  &= 0
\end{align*}

So we obtain:

\begin{align*}
  F(t,x) = E_{t,x}[X_T^3] = x^3 + 3x\sigma^2(T-t)
\end{align*}

We can check the correctness by computing the PDE:

\begin{align*}
  \frac{\partial F(t,x)}{\partial t} + \frac{\sigma^2}{2}\frac{\partial^2F(t,x)}{\partial x^2} = 0 \\
  -3x\sigma^2 + 3x\sigma^2 = 0
\end{align*}

\section{Exercise 3}

We are given the process:

\begin{align*}
  dX(t) &= X(t)dt + dW(t), \; t\geq0 \\
  X(0) &= 0
\end{align*}

The process $Y(t) = e^t \int^t_0 e^sdW(s)$ satisfies this PDE, which can be seen by applying the product rule of Ito's lemma:

\begin{align*}
  dY(t) = d\left(e^t \int^t_0 e^sdW(s)\right) &= \int^t_0 e^sdW(s) de^t + e^t d\left(\int^t_0 e^sdW(s)\right) + de^t d\left(\int^t_0 e^sdW(s)\right) \\
   &= e^t \int^t_0 e^sdW(s) dt + e^t e^{-t} dW(t) + e^t dt e^{-t} dW(t)\\
   &= Y(t) dt + dW(t)
\end{align*}

since $dt dW(t) = 0$.

Also we have $Y(0) = 0$. If we use Lemma 4.15 we can now immediately conclude that $E[Y(t)] = 0$ and $Var[Y(t)] = e^{2t} \int^t_0 e^{-2s}ds = -0.5 e^{2t}(e^{-2t} - 1) = 0.5(e^{2t} - 1)$.

\end{document}
