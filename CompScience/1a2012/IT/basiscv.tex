%%% DOCUMENT SETUP %%%
\documentclass[11pt,a4paper,onecolumn]{article}
\usepackage[english]{babel}

%%% LAYOUT %%%
\usepackage{fullpage}
\usepackage[a4paper]{geometry}
\usepackage[parfill]{parskip}
\usepackage{multicol}
\usepackage{footnote}

%%% GRAPHICS %%%
\usepackage{graphicx}
\usepackage{color}
\usepackage{graphics}
\usepackage{rotating}
\usepackage{subfig}
\usepackage{amsmath}
\usepackage{amssymb}
\usepackage{amscd}
\usepackage{xfrac}
\usepackage{float}
\usepackage{dsfont}
\usepackage{mdwlist}
\usepackage{longtable}

%%% FONT %%%
\usepackage{ifxetex}
\ifxetex
  \usepackage{fontspec}
    \setmainfont{Linux Libertine O}
  \usepackage{xunicode}
  \usepackage{microtype}
\else
  \usepackage[T1]{fontenc}
  \usepackage[latin1]{inputenc}
  \usepackage{times}
  \usepackage{microtype}
\fi

%%% Coding %%%
\usepackage{listings}
\usepackage{pseudocode}

\usepackage{nopageno}

\begin{document}

{\huge \textbf{Curriculum vitae}}

{\Large Jeroen Hofman }\\

{\Large \textbf{Studie en werk} }
\begin{longtable}{l l}
  \begin{tabular}{p{0.10\textwidth} p{0.95\textwidth}}
    1993: & Naar de basisschool. \\
    2001: & Meegespeeld in musical voor afscheid basisschool. \\
    2001: & Naar de middelbare school. \\
    2005 - 2008: & Vaste bijbaan bij de AH: vakkenvullen en afsluiten/openen van zuivel- en groenteafdelingen.\\
    2006: & Eindexamenreis naar Praag. \\
    2007: & Bachelor begonnen in natuurkunde in Utrecht. \\
    2008 \& 2009: & Deelname aan PION (natuurkunde competitie), resp. 6e en 4e plaats: in teamverband werken aan een serie verschillende natuurkunde-opgaven.\\
    2008 - heden: & Bijlessen geven aan leerlingen op het VWO in wiskunde en natuurkunde: helpen met huiswerk maken, uitleggen van theorie en leren plannen en indelen. \\
    2008 - 2009: & Bijlessen geven via studentsplus, inhoud is dezelfde als punt hier boven.\\
    2010: & Cum laude afgestudeerd in natuurkunde: Gedurende de drie jaar bachelor vooral veel theoretisch gericht werk, ook enig modelleer werk.\\
    2010: & Zomerschool gevolgd in Theoretische Fysica: 2 weken intensieve cursussen over uiteenlopende theoretische gebieden met groepsprojecten.\\
    2010: & Master begonnen in Applied Physics in Delft: Veel vakken gericht op toegepaste wiskunde en experimentele fysica, uiteindelijk gestopt na een jaar vanwege motivatieproblemen en het vinden van een interessantere master (zie volgende punt).\\
    2011: & Master begonnen in Computational Science in Amsterdam: projectgericht een probleem aanpakken uit bijvoorbeeld biologie of natuurkunde en dit vertalen naar een computermodel, met een duidelijke wetenschappelijke insteek. Het ontwikkelen van nieuwe modellen en simulaties, affiniteit verkrijgen met software (programmeertalen, applicaties en gedeeltelijk hardware).\\
    2011: & Lid geworden van de onderwijscommissie van informatica: evalueren van vakken en bespreken van zaken die relevant zijn voor de opleiding, zoals structuur en opbouw van vakken, veranderingen in het curriculum etc. \\
    2012: & 2ejaars informatica tutor: begeleiden van studenten in hun werkschema's en problemen met vakken.\\
    2012: & Stage aangeboden gekregen en geaccepteerd bij ING: prijzen van verzekeringsproducten (zogenaamde 'Variable Annuities') door middel van het bouwen van een simulatie dat de prijzen modelleert op een GPU (dit moet nog gaan starten in november). \\
  \end{tabular}
\end{longtable}

{\Large \textbf{Familie en omgeving} }
\begin{longtable}{l l}
  2005: & Voor het eerst naar het buitenland op vakantie met familie (Denemarken). \\
  2007: & Eerste grote buitenlandse vakantie met familie (westkust VS). \\
  2009: & Verhuisd naar Utrecht (vanuit ouders in Tiel, na een eerder mislukte poging). \\
  2012: & Grote reis gemaakt (alleen) van 2 maanden naar Australi\'e: backpacken. \\
\end{longtable}

{\Large \textbf{Sport en hobby} }
\begin{longtable}{l l}
  1994 - 1996: & Muziekles (blokfluit) \\
  1998 - 2001: & Gitaarles: 3 jaar lang samen spelen en oefenen met een studiegenoot. \\
  2000 - 2003: & Lid van zwemvereniging: 3 jaar trainingen en ook gedeeltelijk wedstrijden. \\
  2002 - 2009: & Keyboardles: individuele begeleiding (drie verschillende diploma's gehaald + theorie). \\
  2011 - heden: & Wekelijkse spinclass: Groepsles van 1 uur per week. \\
\end{longtable}

\section*{Reflectie}
Het eerste wat me opvalt aan mijn basisCV is dat ik weinig losse activiteiten heb gedaan, de voorbeelden die worden aangehaald in de toelichting bij de opdracht zijn voor mij nauwelijks aanwezig. Ik heb wel projectmatig werk gedaan, maar dit was altijd sterk ge\"integreerd in de studie en nooit organisatorisch. Naast mijn studie zijn ook bijbaantjes vaak studiegerelateerd.

Een rode draad zou dan ook het helpen van mensen met studiegerelateerde problemen zijn en individueel begeleiden is iets wat ik ook goed kan. Mijn praktische ervaring, in termen van leidinggeven of organiseren, is erg gering, mijn kennis en vaardigheden zijn vooral theoretisch van aard. Dit is echter ook de reden dat ik binnenkort ga beginnen bij ING, om in een praktisch gerichte omgeving te werken. De hoofdactiviteiten zijn dan ook individueel studeren of projectgericht studeren op het studievlak en individueel begeleiden op het werkvlak. Om die reden zijn vaardigheden zoals nauwkeurig werken, concentratievermogen en analytisch vermogen zeer goed ontwikkeld, maar sociale interactie/vaardigheden en groepsinteractie in het bijzonder zijn nog niet goed ontwikkeld.

Een keuze maken tussen verschillende masters was een activiteit die mij minder goed af ging. Gedurende mijn bachelor ben ik vaak van mening veranderd en tegen het einde heb ik een master gekozen wat op dat moment mij het interessantst leek. In deze zin ben ik erg onzeker en is dat zeker een punt waarin ik me kan ontwikkelen. Daarnaast heb ik geprobeerd om op kamers te gaan, maar mislukte dat bij de eerste poging, omdat ik niet goed kon wennen aan een nieuwe omgeving. De tweede poging slaagde wel. Dit is direct gerelateerd aan aanpassingsvermogen, wat bij mij niet heel goed ontwikkeld is. Daarnaast heb ik voor mijn stage bij ING informeel gesolliciteerd. Ik voelde mij echter niet op mijn gemak en g\"intimideerd en had niet het gevoel een goed verhaal te kunnen vertellen waarom ik geschikt zou zijn voor het project dat ze in gedachten hadden. Dit kwam ook voort uit onzekerheid en een sociale situatie waarin ik me niet op mijn gemak voelde.

Mijn bachelorstudie in Utrecht was een succes, omdat ik het gedurende 3 jaar erg naar mijn zin heb gehad. De theoretische insteek en de regelmatige werkdruk zijn dingen waar ik mij goed in kan vinden omdat ik goed plan en erg nauwkeurig ben in mijn werk. Daarnaast had ik ook een groep vrienden met wie ik samenwerkte wat me erg goed beviel. Mijn backpack reis naar Australi\"e was ook een succes, ik heb het naar mijn zin gehad en het was ook geen probleem om alleen te reizen, ik heb ook vrij veel het contact opgezocht met andere mensen wat ook meestal goed lukte. Ik was redelijk zelfverzekerd gedurende de reis en ik had ook geen last van heimwee of iets dergelijks. Over het geven van bijlessen ben ik over het algemeen ook tevreden, ik krijg meestal positieve feedback over mijn geduld en het uitleggen van (meestal) wiskunde op een begrijpelijk niveau.

Ik ben trots op mijn reis naar Australi\"e, omdat ik nog nooit zo lang en zo ver alleen weg was geweest en dit is goed verlopen, zie ook hierboven. Daarnaast ben ik ook trots op dat ik een eigen leven heb opgebouwd in de laatste 3 jaar in Utrecht met een redelijk stabiele groep contacten en ook enkele activiteiten naast mijn studie.

\end{document}

